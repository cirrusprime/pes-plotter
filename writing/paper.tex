\documentclass{article}

\usepackage{fixltx2e}
\usepackage[margin=1in]{geometry}

\newcommand{\tsup}{\textsuperscript}
\newcommand{\tsub}{\textsubscript}

% Path for all graphics
%\graphicspath{{/home/kevin/Dropbox/CCQC/Research/Current_Projects/MetalOxo/2_Draft2/images/}}

\begin{document}

\title{LLaMa15: Automatic Generation of \\ Publication Quality Plots of Potential Energy Surfaces}
\author{Kevin B. Moore III}
\maketitle

\section{Introduction}

A common research goal of scientists in fields such as computational chemistry and spectroscopy, is to map out the potential energy surfaces of various physicochemical processes. This information is often presented in publications and presentations as a singular plot, showing the relative energies of chemical species along some reaction coordinate. For research groups who frequently create such plots, the ability to efficiently generate the plots - especially in a way that meets print standards of publications - is a useful tool.

In many cases, researchers can use more standard graphical editors to draw out the surface. For instance, one could use Illustrator or Photoshop, or one can turn to ChemDraw which is familiar to chemists. Alternatively, one can turn to a script based plotting programs, such as Wolfram Mathematica, MAPLE, or MATLAB. Each of these programs can be quite powerful and can generate a variety of plots. However, each of these approaches offer significant limitations. Two prominent issues deal with the cost of many of these programs - which may prevent certain certain groups from obtaining access to each of the program. Second, is the steep learning curve often associated with learning how to use each of the programs to make even simple plots - regardless of whether they are graphically based or script based. Even for groups and individuals who have access and proficiency in each of these programs, there lies the secondary issue of generating a nicely scaled plot, and more importantly to make changes to the plot without having to start over.

We wish to offer a program that is simple to use, while still being robust enough to directly handle many plotting needs (or at least the ability). To this end, we have written a plotting program, affectionately named LLaMa15, after the code authors. We have built the program with the Python 3.0 language and the Matplotlib module to handle to explicit generation of all plots. This platform offers allows us to create a program that is free and accessible to all users, regardless of their choice of operating system. 

\section{Features}

Included with the code is a user's manual which details the full installation and usage of the program, as well as all of the key features. Here, however, we provide a summary of how the program works.

A key component of the program is configuring the input used by the program to generate the plots. Without ease of use, the program loses a lot of the reason for being. And so, a heavy emphasis has been placed on a simple input structure.   

\section{Future Directions}

Efficient plotting of potential surfaces has served as the impetus for the construction of this program. However, the modular nature of the program should allow for a number of useful plotting features to be added into the program. 

\section{Availabilty}

Our code is offered free for download at github.com (or ccqc.com?). Instructions obtaining and installing each of the neccessary modules to run the program. The code has been 'precompiled' files, however, the source code has been included for this who wish to amend the software to their liking. We encourage anyone to contact us at email to report bugs or request features. We ask only that publications which use plots  generated by our code, cite this paper.

\section{Acknowledgments}

We would like to thank several members of the CCQC for testing out the code, pointing out bugs and suggesting design changes.

% References
%\bibliographystyle{ieeetr}
%\bibliography{2_MetalOxoRefs}


\end{document}